\documentclass[utf8,xcolor=table]{beamer}

\usepackage[T2A]{fontenc}
\usepackage[utf8]{inputenc}
\usepackage[english,russian]{babel}
\usepackage{minted}
\usepackage{ulem}
\usepackage{cmap}
\usepackage{multirow}

\hypersetup{colorlinks,linkcolor=blue,urlcolor=blue}

\mode<presentation>{
	\usetheme{CambridgeUS}
}

\renewcommand{\t}[1]{\ifmmode{\mathtt{#1}}\else{\texttt{#1}}\fi}

\title{Функциональное программирование}
\author{Егор Суворов}
\institute[СПбГУ]{Курс <<Парадигмы и языки программирования>>, группа 18.Б09-пу}
\date[17.04.2019]{Среда, 17 апреля 2019 года}

\setlength{\arrayrulewidth}{1pt}

\begin{document}

\begin{frame}
\titlepage
\end{frame}

\begin{frame}{План занятия}
	\tableofcontents
\end{frame}

\input{01-intro-01-python}
\input{01-intro-02-haskell}
\input{02-haskell-01-recursion}
\input{02-haskell-02-calls}
\input{02-haskell-03-polymorph}
\input{02-haskell-04-lazy}
\input{02-haskell-05-summary}
\subsection{Грабли и плюшки}

\begin{frame}
	\tableofcontents[currentsection,currentsubsection]
\end{frame}

\begin{frame}[fragile]{Классы типов}
	\begin{itemize}
		\item Иногда хочется функцию полиморфную, но с ограничением на тип.
		\item Например, \t{min} должен уметь сравнивать свои аргументы.
		\item Набор свойств, которыми должен обладать тип, называют \textit{классом типа}.
		\item Подробнее разберём на следующем занятии.
		\item То, что идёт перед \t{=>} в типе в Haskell "--- это как раз ограничения на классы:
\begin{minted}{haskell}
min :: Ord a => a -> a -> a
\end{minted}
		\item \t{a} "--- любой тип, лежащий в классе \t{Ord}.
		\item Не путать с классами из ООП!
		\item Тут <<класс>> означает <<множество>>, как в математике.
	\end{itemize}
\end{frame}

\begin{frame}{Очень строгая типизация}
	% 08-01-types.hs
	\begin{itemize}
		\item Есть разные типы для вещественных и целых чисел.
		\item Оператор \t{==} между разными типами чисел не определён.
		\item Но тестом \t{2 == 2.0} это не поймать.
		\item Тип числовой константы определяется в момент компиляции: это будет либо вещественное число (класс \t{Fractional}), либо целое (класс \t{Integral}), либо любое (класс \t{Num}).
		\item Для конвертации между \t{Fractional} и \t{Integral} можно использовать \t{fromIntegral} и \t{round}.
	\end{itemize}
\end{frame}

\begin{frame}[fragile]{Каррирование}
	\begin{itemize}
		\item Функция от двух аргументов $f(x, y)$ "--- то же самое, что функция от одного аргумента $f'(x)$,
			которая возвращает функцию от одного аргумента $g_x(y)=f(x, y)$.
		\item Такой переход называется \textit{каррированием}
		\item С точки зрения Haskell, функции типов \t{a -> b -> c} и \t{a -> (b -> c)} неотличимы.
		\item Все функции в Haskell "--- от одного аргумента, остальное "--- сахар.
		\item Из-за этого частичное применение и работает: \t{(3+)} и \t{(+) 3} равны.
		\item Можно явно написать функцию от двух аргументов:
\begin{minted}{haskell}
s (a, b) = a + b
s' = uncurry (+)
\end{minted}
	\end{itemize}
\end{frame}

\begin{frame}[fragile]{Способы определения функций}
	\begin{itemize}
		\item
\begin{minted}{haskell}
addTen x = x + 10
addTenToAll x = map addTen x
\end{minted}
		\item
\begin{minted}{haskell}
addTen = (+10)
addTenToAll = map f
\end{minted}
	\end{itemize}
\end{frame}

\begin{frame}[fragile]{Композиция функций}
	Точка "--- оператор композиции функций:
\begin{minted}{haskell}
(.) :: (b -> c) -> (a -> b) -> (a -> c)
-- Приоритет низкий
((+5) . (*3)) 10
\end{minted}
	Читать надо справа налево:
\begin{minted}{haskell}
(concat . map words . lines) "1 2\n10 11\n" 
\end{minted}
	Композиции функций в каком-то смысле лучше, чем свои вспомогательные функции или явные вызовы.
\end{frame}

\begin{frame}{Унарный минус}
	\begin{itemize}
		\item Единственный унарный оператор в Haskell.
		\item Захардкожен костылями.
		\item Ломает красоту, потому что \t{(-3)} надо интерпретировать как унарный минус, а не как оператор \t{-} с зафиксированным операндом.
		\item А \t{((-)3)} надо интерпретировать, как оператор \t{(-)} с зафиксированным первым (левым) операндом.
		\item Поэтому зафиксировать правый аргумент у бинарного минуса никак нельзя, кроме лямбда-функций.
	\end{itemize}
\end{frame}

\input{02-haskell-07-debug}
\input{03-bonus}
\input{04-links}

\end{document}
