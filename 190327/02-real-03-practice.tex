\subsection{Практические последствия}

\begin{frame}
	\tableofcontents[currentsection,currentsubsection]
\end{frame}

\begin{frame}[fragile]{Хранятся только старшие знаки}
\begin{minted}{python}
for a in [2.0 ** x for x in [10, 20, 40, 60, 1000]]:
    print(a == a + 1, a == a * (1 + 2 ** (-40)), a + 1 - a)
\end{minted}
	В Python у нас тип двойной точности, поэтому до $2^{40}$ единица ещё будет играть роль, а вот после "--- нет.

	Складывать/вычитать числа разных порядков "--- плохо.
	Сильно теряется точность.
\begin{minted}{python}
print(1000 + 1e100 - 1e100)
print(1e100 + 1000 - 1000)
\end{minted}
	Одного порядка "--- нормально.
\end{frame}

\begin{frame}[fragile]{Ошибка может копиться}
\begin{minted}{python}
from random import shuffle
a=[123456 * 2 ** x for x in range(0, 200)]
print(format(sum(a), ".20e"))  # Точный результат
a = list(map(float, a))
for _ in range(10):
    shuffle(a)
    print(format(sum(a), ".20e"))
\end{minted}
	\begin{itemize}
		\item
			Так как результаты округляются на каждом шаге, результат может зависеть от порядка вычислений, размеров чисел и фазы луны.
		\item
			Пример, где всё может быть плохо "--- метод Гаусса решения систем линейных уравнений (подробнее расскажут на алгоритмах).
		\item
			Уже для системы $15 \times 15$ можно подобрать матрицу, на которой точность теряется катастрофически быстро.
	\end{itemize}
\end{frame}

\begin{frame}{Оптимизатор}
	\begin{center}
		\includegraphics[scale=0.4]{optimizer-game.jpg}
	\end{center}
\end{frame}

\begin{frame}{Оптимизатор}
	\begin{itemize}
		\item
			Часто процессор поддерживает числа большего размера, чем double-precision.
		\item
			Тогда точность может в некоторых местах случайно возрастать, а оптимизатор об этом не догадывается.
		\item
			Из-за этого <<побитово равные>> числа могут \href{http://codeforces.com/blog/entry/1059}{оказаться различными}:
			в одном месте сравнили куски памяти, а в других местах "--- лежащие в процессоре числа большей точности.
		\item
			Аналогично, можно случайно (не) увидеть какую-то маленькую погрешность.
	\end{itemize}
\end{frame}

\begin{frame}[fragile]{Десятичные дроби неточны}
\begin{minted}{python}
print(0.1 + 0.2)  # Классика
print(format(0.1 + 0.2, ".100f"))
print(format((0.1 + 0.2) * 2 ** 52, ".100f"))
for n in [1000, 1024]:
    fail, total = 0, 0
    for a in range(n):
        for b in range(n):
            if (a + b) / n != a / n + b / n:
                fail += 1
            total += 1
    print(fail, total, fail / total)
\end{minted}
\end{frame}

\begin{frame}{Что делать?}
	\begin{itemize}
		\item
			Не использовать вещественные числа.
		\item
			Особенно на контестах.
		\item
			Особенно у Серёжи.
		\item
			Если очень надо "--- готовиться к неточности и допускать погрешность при \textit{любых} сравнениях (см. оптимизатор):
			\begin{center}
				\begin{tabular}{c|c}
					Было & Стало \\\hline
					\t{a == b} & \t{abs(a - b) < eps} \\
					\t{a <= b} & \t{a <= b + eps}  \\
					\t{a < b} & \t{a < b - eps} \\
				\end{tabular}
			\end{center}
			Тут мы считаем, что числа равны, если отличаются не более, чем на \t{eps}.
		\item
			Следить за порядком операций и избегать сложения/вычитания чисел разного порядка.
		\item
			Иногда можно преобразовывать формулы: $x^2-y^2=(x-y)(x+y)$.
	\end{itemize}
\end{frame}

\begin{frame}{Замечания про сравнения}
	\begin{itemize}
		\item
			Можно сравнивать числа ещё аккуратнее, если смотреть на относительную погрешность, а не абсолютную.
			Или если надо сравнивать в программе и большие числа, и маленькие.
		\item
			Подбор \t{eps} и правильного порядка операций "--- отдельное искусство.
		\item
			Обычно на контестах берут \t{eps} от $10^{-5}$ до $10^{-15}$; мой любимый "--- $10^{-8}$.
		\item
			При желании может строго теоретически обосновать правильный \t{eps}.
		\item
			\pause
			Да и неправильный тоже :(
	\end{itemize}
\end{frame}

\begin{frame}[fragile]{Странные вычисления}
	Осторожно с \t{NaN}: любое вычисление с ним немедленно породит \t{NaN}, а сравнения с ним всегда выдают \t{false}:
\begin{minted}{python}
from math import sqrt
print(2 ** 1000 / 2 ** (-1000))
print(-2 ** 1000 / 2 ** (-1000))
print(sqrt(0), sqrt(2 ** (-1000)))
print(0 * float('inf'))
print(1 + float('nan'))
print(float('nan') == float('nan'))
\end{minted}
	В C++ его ещё можно получить, взяв корень из отрицательного числа или $\frac{0}{0}$ (Python может кинуть исключение).

	Лайфхак: когда извлекаете корень, не извлеките его случайно из \t{-eps}.
	Для этого стоит писать \t{sqrt(max(x, 0))}, если $x$ может быть близок к нулю.
\end{frame}
