\documentclass[utf8,xcolor=table]{beamer}

\usepackage[T2A]{fontenc}
\usepackage[utf8]{inputenc}
\usepackage[english,russian]{babel}
\usepackage{tikz}
\usetikzlibrary{shapes,arrows}
\usepackage{dot2texi}
\usepackage{minted}
\usepackage{ulem}
\usepackage{cmap}
\usepackage{multirow}
\usepackage{bigdelim}

\hypersetup{colorlinks,linkcolor=blue,urlcolor=blue}

\mode<presentation>{
	\usetheme{CambridgeUS}
}

\newcommand{\svgimg}[1]{
  \begin{center}
	\includegraphics[width=\textwidth,height=0.8\textheight,keepaspectratio]{#1.pdf}
  \end{center}
}
\renewcommand{\t}[1]{\ifmmode{\mathtt{#1}}\else{\texttt{#1}}\fi}

\title{Базы данных и язык SQL}
\author{Егор Суворов}
\institute[СПбГУ]{Курс <<Парадигмы и языки программирования>>, группа 18.Б09-пу}
\date[08.05.2019]{Среда, 8 мая 2019 года}

\setlength{\arrayrulewidth}{1pt}

\begin{document}

\begin{frame}
\titlepage
\end{frame}

\begin{frame}{План занятия}
	\tableofcontents
\end{frame}

\section{Основы основ}
\subsection{Что такое СУБД}

\begin{frame}
	\tableofcontents[currentsection,currentsubsection]
\end{frame}

\begin{frame}{Постановка задачи}
	\begin{itemize}
		\item Пусть мы пишем приложения для учёта товаров в магазине.
		\item Надо знать:
			\begin{enumerate}
				\item Какой товар есть на складе и витринах.
				\item Где он лежит.
				\item По какой цене товар закуплен (могут быть разные партии).
				\item По какой цене товар сейчас продаётся.
				\item Какие покупки были сделаны (что куплено вместе, на какую сумму, в какое время).
			\end{enumerate}
		\item Возможные события:
			\begin{enumerate}
				\item Приехала поставка со склада.
				\item Касса пробила чек "--- совершена покупка.
			\end{enumerate}
		\item Надо, чтобы приложение сохраняло состояние между перезапусками.
		\item Вопрос: как это сделать?
	\end{itemize}
\end{frame}

\begin{frame}{Усложнения}
	\begin{itemize}
		\item Запуск должен быть быстрый
		\item Данные могут не помещаться в память
		\item Может случайно отключаться электричество
		\item Десять касс и два компьютера в разных концах склада.
		\item Хочется получать обновления <<как только так сразу>>
		\item Иногда может теряться связь между кассами и складом
		\item Может меняться формат хранения (например, добавили бонусы за товар)
		\item Кассы не должны уметь как угодно менять данные
	\end{itemize}
\end{frame}

\begin{frame}[t]{СУБД}
	\begin{itemize}
		\item \textit{Система управления базами данных} (СУБД) "--- это сервис, который умеет хранить данные \textit{произвольной структуры}
			(в определённых рамках, конечно, не совсем бессистемные).
		\item \textit{База данных} "--- это описание данных \textbf{и их структуры}, которые хранятся в СУБД.
		\item \textit{Реляционные БД} "--- это классика (существуют с 80-х годов), их и будем изучать.
		\item Обычно запросы к реляционным СУБД формулируются на декларативном языке SQL
			(Structured Query Language).
		\item Примеры реляционных: MySQL, MariaDB, Oracle, MS SQL, Sqlite.
		\item Примеры нереляционных (нынче модно): MongoDB, Redis, Memcached, Cassandra.
	\end{itemize}
\end{frame}

\begin{frame}[t]{В чём плюсы}
	\begin{itemize}
		\item Можно выделить один (или несколько) больших серверов под хранение обработку данных; для всех приложений в организации
		\item Автоматически получаем единый контроль доступа, бэкапы, масштабирование и прочие плюшки
		\item SQL все знают и могут делать запросы к БД напрямую (для отладки/отчётов)
		\item Сложнее посадить баг в коде и уронить приложение
	\end{itemize}
\end{frame}

\begin{frame}{В чём минусы}
	\begin{itemize}
		\item
			Мы отдаём контроль за скоростью выполнения и потреблением памяти в руки СУБД
			(как и при любой абстракции).
			Это обычно приемлемый компромисс.
		\item
			Приложение сложнее запустить: нужно настроить СУБД, что обычно занимает несколько шагов.
			В нестадартных ситуациях "--- больше.
		\item
			Иногда приложение требует слишком хитрую настройку СУБД (например, для корректной работы
			с не-латиницей и датами).
		\item
			Многие инструменты заточены под промышленные решения и имеют слишком много рычажков и кнопок
			для простых целей.
	\end{itemize}
\end{frame}

\begin{frame}{Встраиваемые СУБД}
	\begin{itemize}
		\item Самая известная встраиваемая СУБД "--- sqlite.
		\item Предназначена не для сетевого доступа, а для использования в рамках одной конкретной программы.
		\item Её можно просто вкомпилировать в своё приложение, не требуется никакой настройки.
		\item sqlite хранит каждую БД в отдельном файле определённого формата (последний "--- sqlite3).
		\item Формат sqlite3 один на все приложения, можно даже залезть в чужие БД и посмотреть.
		\item Занимает мало места в скомпилированном приложении.
		\item
			Используется \href{http://www.sqlite.org/famous.html}{во многих приложениях}:
			под Android, в Firefox, в Chrome, в клиенте Dropbox\footnote{ищите файлы \t{.db}, \t{.sqlite}, \t{.sqlite3}}...
		\item Задание: скачайте файл \t{example.sqlite3}
	\end{itemize}
\end{frame}

\begin{frame}{Реляционные СУБД на практике}
	\begin{itemize}
		\item СУБД хранит одну или несколько независимых БД (баз данных).
		\item Каждая БД "--- это набор \textit{таблиц} (<<отношений>>), которые содержат данные.
		\item Таблица имеет фиксированный набор столбцов с названиями и типами.
		\item Фиксированный в каждый момент времени; вообще столбцы можно добавлять, менять, удалять, хоть это и сложные для СУБД операции.
		\item В таблице лежит неупорядоченный набор строк с данными.
		\item На столбцы (или их группы) могут накладываться дополнительные ограничения (например, <<все значения в столбце различны>>).
	\end{itemize}
\end{frame}

\subsection{Использование СУБД}

\begin{frame}
	\tableofcontents[currentsection,currentsubsection]
\end{frame}

\begin{frame}{Анонс домашнего задания}
	\begin{itemize}
		\item Вам будет выдан файл с SQL-запросами, которые создают таблицы со странами (структуру разберём) и заполняют их данными.
		\item Вам нужно написать несколько SQL-запросов \t{SELECT}, которые что-то вычисляют.
		\item Тестировать можно на созданных тестовых данных.
		\item Как именно тестировать "--- сейчас покажу.
	\end{itemize}
\end{frame}

\begin{frame}{Консольная утилита}
	\begin{itemize}
		\item Называется sqlite3. Это просто программа, которая умеет выполнять SQL-запросы на БД sqlite.
		\item По умолчанию создаёт пустую БД в памяти.
		\item Можно попросить открыть существующую БД в файле (или создать новый файл).
		\item При помощи перенаправления может выполнять SQL из файла.
		\item SQL-запрос должен заканчиваться точкой с запятой.
	\end{itemize}
\end{frame}

\begin{frame}{Графическая утилита}
	\begin{itemize}
		\item Я выбрал \href{http://sqlitebrowser.org/}{DB Browser for SQLite}.
		\item Иногда проще смотреть на таблице в графической оболочке, чем в консоли.
		\item Может открывать файлы с БД, все изменения идут в памяти.
		\item Можно откатывать изменения кнопкой <<Revert Changes>> до последнего сохранения.
		\item Можно сохранять изменения в файл кнопкой <<Write Changes>>.
		\item Показывает таблицы, их структуру, позволяет выполнять произвольные запросы.
	\end{itemize}
\end{frame}

\input{01-intro-03-simple-sql}
\subsection{Бонус: доступ из кода}

\begin{frame}[fragile]{Python}
\begin{minted}{python}
with sqlite3.Connection("literacy.sqlite3") as db:
  cursor = db.execute("SELECT * FROM Country LIMIT 3")
  print(cursor.description)
  print(list(cursor))
  print(list(cursor))  # Что-нибудь выведет?
\end{minted}
	\begin{itemize}
		\item Терминология очень похожа во всех языках и СУБД.
		\item Обычно в языке есть стандартный интерфейс общения с любыми СУБД.
			А \textit{драйвер} СУБД реализует этот интерфейс в языке.
		\item Сначала мы устанавливаем \textit{соединение} с СУБД.
		\item Результатом запроса является \textit{курсор} "--- это такой итератор по строчкам запроса.
		\item Что возвращают запросы, кроме \t{SELECT} "--- зависит от СУБД.
		\item
			Иногда считается, что не запрос возвращает курсор, а надо
			сначала создать курсор, а потом в нём выполнить запрос.
	\end{itemize}
\end{frame}

\begin{frame}[t,fragile]{SQL-инъекции}
	Пусть есть таблица с полями: владелец текста, его название, содержимое.
	Код для доступа к базе, выполняется на сервере:
\begin{minted}{python}
with sqlite3.Connection("sql-injection.sqlite3") as db:
  key = input('Text key: ')
  cursor = db.execute("""SELECT * FROM Text
                         WHERE owner='user' AND key='{}'"""
                         .format(key))
  print(list(cursor))
\end{minted}
	\only<1-2>{
	В чём проблема?
	\only<2>{
	\begin{center}
		\begin{tabular}{ll}
			Значение: & \t{key1} \\\hline
			Было:  & \t{SELECT ... WHERE owner='user' AND key='\{\}'} \\\hline
			Стало: & \t{SELECT ... WHERE owner='user' AND key='key1'} \\
		\end{tabular}
	\end{center}
	}
	}
	\only<3-4>{
	К коллайдеру!
	\only<4>{
	\begin{center}
		\begin{tabular}{ll}
			Значение: & \t{' OR ''='} \\\hline
			Было:  & \t{SELECT ... WHERE owner='user' AND key='\{\}'} \\\hline
			Стало: & \t{SELECT ... WHERE owner='user' AND key='' OR ''=''} \\
		\end{tabular}
	\end{center}
	Упс.
	}
	}
\end{frame}

\begin{frame}{Классический комикс}
	\begin{center}
		\includegraphics[scale=0.5]{xkcd-327.png}
	\end{center}
\end{frame}

\begin{frame}[fragile]{А как правильно?}
\begin{minted}{python}
with sqlite3.Connection("sql-injection.sqlite3") as db:
  key = input('Text key: ')
  cursor = db.execute("""SELECT * FROM Text
                         WHERE owner='user'
                           AND key=?""", [key])
  print(list(cursor))
\end{minted}
	Теперь драйвер базы данных знает, что \t{key} "--- это значение от пользователя,
	которое надо \textit{заэкранировать}:
	\begin{center}
		\begin{tabular}{ll}
			Значение: & \t{' OR ''='} \\\hline
			Было:  & \t{... WHERE owner='user' AND key=?} \\\hline
			Стало: & \t{... WHERE owner='user' AND key='\textbackslash' OR  \textbackslash'\textbackslash'=\textbackslash''} \\
		\end{tabular}
	\end{center}
	Независимо от того, какой код мы напишем, SQL-инъекции не случится.

	Мораль: никогда не собирайте SQL-запрос руками из переменных.
\end{frame}

\section{Запросы посложнее}
\begin{frame}
	\tableofcontents[currentsection]
\end{frame}

\subsection{Группировка строк}
\begin{frame}
	\tableofcontents[currentsection,currentsubsection]
\end{frame}

\begin{frame}[fragile]{GROUP BY "--- картинка}
	\begin{center}
		\begin{minipage}{0.4\textwidth}
\begin{minted}{sql}
SELECT SUM(Pop), GovForm
FROM Country
WHERE ...
GROUP BY GovForm
HAVING ...
\end{minted}
		\end{minipage}
\vspace{10pt}

		\begin{tabular}{rllclll}
			\hline
			& Pop & GovForm & & SUM(Pop) & GovForm & \\\hline
			\multirow{6}{*}{$\xrightarrow{\t{WHERE}}$} & 618 & Republic & \rdelim\}{3}{20pt}[$\longrightarrow$] & \multirow{3}{*}{1134} & \multirow{3}{*}{Republic} & \multirow{6}{*}{$\xrightarrow{\t{HAVING}}$}\\
			& 455 & Republic \\
			& 61 & Republic \\
			& 650 & Monarchy & \rdelim\}{2}{20pt}[$\longrightarrow$] & \multirow{2}{*}{17425} & \multirow{2}{*}{Monarchy} & \\
			& 17364 & Monarch \\
			& \vdots & \vdots & & \vdots & \vdots & \\
		\end{tabular}
	\end{center}
\end{frame}

\begin{frame}{GROUP BY}
	\begin{itemize}
		\item Полезно, когда мы хотим посчитать какую-то статистику по подмножествам строк.
		\item Каждая агрегирующая функция работает только внутри группы.
		\item Например, суммарное население стран с разным политическим строем.
		\item Можно группировать по нескольким полям.
		\item Условие \t{WHERE} применяется до группировки.
		\item \t{ORDER BY} применяется после (так как группировка от порядка не зависит).
		\item В \t{SELECT} можно использовать только агрегирующие функции и колонки, по которым сделана группировка
			(иначе неясно, из какой строки выбирать).
			Некоторые СУБД ругаются, некоторые делают что-то.
		\item Можно дополнительно отфильтровать результаты после группировки при помощи \t{HAVING}.
	\end{itemize}
\end{frame}

\begin{frame}{Упражнения}
	Посчитайте:
	\begin{enumerate}
		\item Среднюю площадь страны в зависимости от формы правления.
		\item Средний квадрат площади страны в зависимости от формы правления.
		\item Количество стран с площадью порядка миллиона, порядка двух миллионов, и так далее.
	\end{enumerate}
\end{frame}

\subsection{Подзапросы}
\begin{frame}
	\tableofcontents[currentsection,currentsubsection]
\end{frame}

\begin{frame}{Подзапросы-1}
	\begin{enumerate}
		\item
			\begin{itemize}
				\item Задача: хотим для города найти население страны, в которой он расположен.
				\item В таблице городов есть только код страны, без населения.
				\item Можно сделать подзапрос в условии \t{WHERE}.
			\end{itemize}
		\item
			\begin{itemize}
				\item Задача: хотим найти средний уровень самой грамотной страны по годам.
				\item Надо две агрегатных функции: сначала группируем по годам (чтобы найти победителя), а потом берём среднее.
				\item Можно сделать подзапрос в \t{FROM} (ведь результат SELECT "--- тоже таблица, которую можно назвать).
			\end{itemize}
	\end{enumerate}
\end{frame}

\begin{frame}{Подзапросы-2}
	\begin{itemize}
		\item Задача: хотим вывести информацию по странам плюс население самого большого города.
		\item Эта информация лежит в двух разных таблицах.
		\item SQL позволяет делать \t{SELECT FROM} сразу из нескольких таблиц (получается декартово произведение).
		\item Можно взять декартово произведение городов и стран, оставить только соответствующие, а по оставшимся взять агрегирующую функцию.
		\item Если имена колонок в разных таблицах совпадают, надо явно указывать, к какой мы обращаемся.
		\item Вообще лучше всегда явно указывать, из какой таблицы мы берём колонку, если таблиц несколько.
	\end{itemize}
	Такого сорта выборки происходят очень часто, в реляционной алгебре они зовутся <<соединениями>> (join).
\end{frame}

\subsection{Соединения}
\begin{frame}
	\tableofcontents[currentsection,currentsubsection]
\end{frame}

\begin{frame}{Соединения}
	\begin{itemize}
		\item Можно считать синтаксическим сахаром для взятия подмножества декартова произведения.
		\item Лучше отражает суть происходящего и проще читается.
		\item После \t{ON} может быть произвольное условие (это круче, чем в реляционной алгебре).
		\item Обычно там ставят условие <<номер страны в первой таблице равен номеру страны во второй таблице>>.
	\end{itemize}
\end{frame}

\begin{frame}[fragile]{Соединения "--- картинка}
	\svgimg{join-01}
\end{frame}

\begin{frame}{Ключи и соединения-1}
	\begin{itemize}
		\item Обычно сущностей в базе много и они как-то связны отношениями (<<каждый город лежит ровно в одной стране>>).
		\item Не хочется дублировать информацию в разных таблицах (место занимает, изменять сложно).
		\item Поэтому информация о стране/городе отдельно.
		\item А запрос <<получи объект по вот этому отношению>> возникает.
		\item
			Так как свойства объектов часто меняются, то обычно каждому объекту выдают \textit{первичный ключ},
			по которому его можно опознать.
			Обычно это просто какое-то число (возможно, с автоинкрементом).
	\end{itemize}
\end{frame}

\begin{frame}{Ключи и соединения-2}
	\begin{itemize}
		\item Тогда связи вроде <<в какой стране лежит город>> "--- это просто столбец <<номер страны>> в таблице с городом.
		\item Такой столбец называют \textit{внешним ключом}.
		\item Это даже можно отразить в структуре таблицы (\t{REFERENCES}).
		\item И ещё можно указать, что делать при удалении того объекта, куда мы ссылаемся (\t{ON DELETE CASCADE}).
	\end{itemize}
\end{frame}

\begin{frame}{Упражнения}
	\begin{enumerate}
		\item Вывести для каждой страны максимальный уровень её грамотности за все года.
		\item Вывести для каждой страны номер города-столицы (см. таблицу \t{Capital}).
		\item Вывести для каждой страны название города-столицы (потребуется два \t{JOIN}).
	\end{enumerate}
\end{frame}

\begin{frame}{Соединения, NULL, отсутствие значений}
	\begin{enumerate}
		\item Посчитаем количество стран (239).
		\item Теперь для каждой страны посчитаем количество городов.
		\item Теперь посмотрим, сколько строк получили в результате "--- 232.
	\end{enumerate}
	\begin{itemize}
		\item А всё потому что есть страны, в которых городов нет "--- про них в таблице \t{City} просто нет информации.
		\item И соединение не поможет "--- страна без городов отфильтруется.
		\item Но есть \t{LEFT (OUTER) JOIN} "--- он обязуется добавить в соединение все строчки из <<левой>> таблицы
			(а если не нашлось соответствующих строк, то поля второй в строчке соединения будут \t{NULL}).
		\item С ним надо быть осторожным "--- потому что теперь в строчках соединения могут оказаться \t{NULL},
			которые, скорее всего, не надо учитывать (\t{COUNT(*)}, например, их учтёт).
		\item Ещё бывают аналогичные \t{RIGHT JOIN} и \t{FULL JOIN} (SQLite их не поддерживает).
	\end{itemize}
\end{frame}

\begin{frame}[fragile]{Соединения "--- картинка}
	\svgimg{join-02}
\end{frame}

\begin{frame}{Классическая шутка}
	\begin{center}
		\includegraphics[scale=0.4]{join-two-tables.png}
	\end{center}
\end{frame}

\subsection{Объединения результатов запросов}
\begin{frame}
	\tableofcontents[currentsection,currentsubsection]
\end{frame}

\begin{frame}{UNION}
	\begin{itemize}
		\item Хотим вывести названия всех географических объектов из БД.
		\item Делаем два (или больше) \t{SELECT} с одинаковым количеством столбцов в результате.
		\item Если объединяем их через \t{UNION} "--- то в результате не будет одинаковых строк вообще (даже если они были внутри одного \t{SELECT}).
		\item Если объединяем их через \t{UNION ALL} "--- то просто все результаты объединятся вместе.
	\end{itemize}
	На практике:
	\begin{itemize}
		\item Я ни разу не встречал.
		\item Может потребоваться, если в БД есть похожие таблицы (например, \t{UsersUS} и \t{UsersEU}).
	\end{itemize}
\end{frame}


\end{document}
