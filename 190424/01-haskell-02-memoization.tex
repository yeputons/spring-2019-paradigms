\subsection{Мемоизация}

\begin{frame}
	\tableofcontents[currentsection,currentsubsection]
\end{frame}

\begin{frame}[t,fragile]{Ленивые числа Фибоначчи}
\begin{minted}{haskell}
fib 0 = 0
fib 1 = 1
fib n = fib (n - 2) + fib (n - 1)
\end{minted}
	\t{fib 100} работает бесконечно долго.
	Несмотря на то, что \t{fib 10} можно запомнить,
	Haskell не запоминает вообще все вычисления функций в программе
	(а программа "--- это одно большое вычисление).

	Можно обхитрить ленивыми списками:
\begin{minted}{haskell}
fibs = map fib' [0..]
    where
        fib' 0 = 0
        fib' 1 = 1
        fib' n = fibs !! (n - 1) + fibs !! (n - 2)
\end{minted}
	Теперь у нас есть одно ленивое значение (\t{fibs}), которое Haskell
	постоянно помнит.
\end{frame}

\begin{frame}[t,fragile]{Сокращаем код}
	Можно записать даже вот так:
\begin{minted}{haskell}
fib = (map fib' [0..] !!)
    where
        fib' 0 = 0
        fib' 1 = 1
        fib' n = fib (n - 1) + fib (n - 2)
\end{minted}

	А вот так уже не катит:
\begin{minted}{haskell}
fib n = map fib' [0..] !! n
    where
       ...
\end{minted}
	
	Вроде бы\footnote{\href{https://stackoverflow.com/a/3951092/767632}{stackoverflow.com/a/3951092/767632}}
	Haskell вычисляет выражение не больше одного раза \textit{внутри одного вызова лямбда-функции}.
\begin{minted}{haskell}
fib = \n -> map fib' [0..] !! n
\end{minted}
\end{frame}
