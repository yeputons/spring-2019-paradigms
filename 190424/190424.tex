\documentclass[utf8,xcolor=table]{beamer}

\usepackage[T2A]{fontenc}
\usepackage[utf8]{inputenc}
\usepackage[english,russian]{babel}
\usepackage{tikz}
\usetikzlibrary{shapes,arrows}
\usepackage{dot2texi}
\usepackage{minted}
\usepackage{ulem}
\usepackage{cmap}
\usepackage{multirow}

\hypersetup{colorlinks,linkcolor=blue,urlcolor=blue}

\mode<presentation>{
	\usetheme{CambridgeUS}
}

\renewcommand{\t}[1]{\ifmmode{\mathtt{#1}}\else{\texttt{#1}}\fi}

\title{Типы в Haskell}
\author{Егор Суворов}
\institute[СПбГУ]{Курс <<Парадигмы и языки программирования>>, группа 18.Б09-пу}
\date[24.04.2019]{Среда, 24 апреля 2019 года}

\setlength{\arrayrulewidth}{1pt}

\begin{document}

\begin{frame}
\titlepage
\end{frame}

\begin{frame}{План занятия}
	\tableofcontents
\end{frame}

\input{01-haskell-01-high-order}
\input{02-algebraic-01-sum-type}
\input{02-algebraic-02-standard}
\input{02-algebraic-03-usage}
\input{03-classes-01-interfaces}
\input{03-classes-02-extras}

\end{document}
