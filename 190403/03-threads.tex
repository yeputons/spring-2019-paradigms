% !TeX root = 190403.tex
\begin{frame}[t]{Многопоточность}
	\begin{itemize}
	\item В программе могут одновременно выполняться несколько \textit{потоков}
	\item Можно для реального ускорения вычислений (несколько ядер у процессора)
	\item Можно для многозадачности (даже на одном ядре)
	\item Можно передавать сообщения между потоками через \textit{каналы}
	\item Можно использовать общую память (будет потом)
	\end{itemize}
\end{frame}

\begin{frame}[t,fragile]{Каналы}
	MPSC (multiple producer, single consumer):
	\begin{center}
	\begin{forest}
	[,phantom,for descendants={align=center,draw,fit=tight,font=\ttfamily}
		[{fn main()},fit=band
			[{rx.recv() // 10/20},no edge,name=recv1
			[{rx.recv() // 20/10},no edge,name=recv2
			]
			]
		]
		[канал,l*=3,dashed,name=channel]
		[{move || ...},fit=band
			[{tx1.send(10)},name=send1,no edge]
		]
		[{move || ...},name=t2,tier=t2,fit=band
			[{tx2.send(20)},name=send2,no edge]
		]
	]
	\draw [->,dashed] (send1.south) -- (channel.east);
	\draw [->,dashed] (send2.south) -- (channel.east);
	\draw [<-,dashed] (recv1.east) -- (channel.west);
	\draw [<-,dashed] (recv2.east) -- (channel.west);
	\end{forest}
	\end{center}
\end{frame}

\begin{frame}[t,fragile]{Владение каналами}
	\begin{center}
	\begin{forest}
	[,phantom,for descendants={align=center,draw,fit=tight,font=\ttfamily,l=0.2cm,l sep=0.2cm,scale=0.85}
		[{fn main()},name=main,fit=band
			[{(tx: Sender<i32>,\\rx: Receiver<i32>)\\= channel()}
				[{tx\_t1 = tx.clone()},l=1.75cm,s=-1cm,name=tx1a,dashed,edge=dashed,tier=t1
				[{t1 = thread::spawn},name=t1var,no edge
				]
				]
				[{tx\_t2 = tx.clone()},l=3.5cm,name=tx2a,dashed,edge=dashed,tier=t2
				[{t2 = thread::spawn},name=t2var,no edge
				]
				]
			]
		]
		[{move || ...},name=t1,tier=t1,fit=band
			[{tx\_t1: Sender<i32>},name=tx1b]
		]
		[{move || ...},name=t2,tier=t2,fit=band
			[{tx\_t2: Sender<i32>},name=tx2b]
		]
	]
	\draw [->,dashed] (tx1a.east) -- (tx1b.west);
	\draw [->,dashed] (tx2a.east) -- (tx2b.west);
	\draw [->] (t1var.east) -- (t1.west);
	\draw [->] (t2var.east) -- (t2.west);
	\end{forest}
	\end{center}
	Канал жив, пока жив его <<single>> конец (тут "--- \texttt{rx}).
\end{frame}

\begin{frame}[t]{Распараллеливание перебора: до}
	\begin{center}
	\begin{forest}
	[,phantom,for descendants={align=center,draw,fit=tight,font=\ttfamily}
		[?????
			[0????
				[00???
					[000 [\dots]]
					[001 [\dots]]
				]
				[01???
					[010 [\dots]]
					[011 [\dots [01101,red]]]
				]
			]
			[1????,for tree={dashed,edge=dashed}
				[10???
					[100 [\dots [10001,red]]]
					[101 [\dots]]
				]
				[11???
					[110 [\dots]]
					[111 [\dots]]
				]
			]
		]
	]
  \path (current bounding box.west)|-coordinate(l)(!111.south);
  \path (current bounding box.east)|-coordinate(r)(!111.south);
  \draw[blue,thick] (l)--(r);
	\end{forest}
	\end{center}
\end{frame}


\begin{frame}[t]{Распараллеливание перебора: после}
	\begin{center}
	\begin{forest}
	[,phantom,for descendants={align=center,draw,fit=tight,font=\ttfamily,l sep*=0.5,l*=0.5,scale=0.8}
		[?????
			[0????
				[{00???,spawn}
					[00???,edge={->,dashed},tikz={\node [draw,blue,fit=()(!11)(!21)]{};},l*=2
						[000 [\dots]]
						[001 [\dots]]
					]
				]
				[{01???,spawn}
					[01???,edge={->,dashed},tikz={\node [draw,blue,fit=()(!11)(!211)]{};},l*=2
						[010 [\dots]]
						[011 [\dots [{tx01.send(01101)},red,s-=2cm]]]
					]
				]
			]
			[1????
				[{10???,spawn}
					[10???,edge={->,dashed},tikz={\node [draw,blue,fit=()(!111)(!21)]{};},l*=2
						[100 [\dots [{tx10.send(10001)},red,s+=2cm]]]
						[101,for tree={dashed,edge=dashed} [\dots]]
					]
				]
				[{11???,spawn}
					[11???,edge={->,dashed},tikz={\node [draw,blue,fit=()(!11)(!21)]{};},l*=2
						[110 [\dots]]
						[111 [\dots]]
					]
				]
			]
		]
	]
	\end{forest}
	\end{center}
\end{frame}
