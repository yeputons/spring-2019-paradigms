% !TeX root = 190403.tex
\begin{frame}[t]{Что такое ресурс?}
	Что угодно, что надо после использования \textit{закрыть} или \textit{освободить},
	иначе будут неприятности:

	\begin{itemize}
	\item Память
	\item Файлы
	\item Сетевые соединения
	\item Объекты ОС (кисть, окно...)
	\item Memory-mapped-file
	\end{itemize}
\end{frame}

\begin{frame}[t]{Сборщик мусора}
	\begin{itemize}
	\item Можно считать ссылки на каждый объект и освобождать, когда ссылок ноль
	\item Не работает с циклами, но можно использовать как оптимизацию
	\item Бывает \textit{сборщик мусора}: регулярно запускает dfs и удаляет объекты
	\item Когда удаляет "--- неизвестно (может, никогда)
	\item Если нужна детерминированность в закрытии "--- не подходит
	\end{itemize}
\end{frame}

\begin{frame}[t,fragile]{Стандартный шаблон}
	\begin{itemize}
		\item \textit{Захватить} в начале блока, \textit{отпустить}, когда из блока выходим
		\item Нельзя где-то <<сохранить>> открытый ресурс и не забыть закрыть
		\item Python: Менеджеры контекста
		\item Java: try-with-resources
		\item C++: RAII (но тут, правда, можно сохранить)
	\end{itemize}
	Могут быть тонкости, когда ресурсы используют друг друга по цепочке:
\begin{minted}{java}
try (BufferedReader r = new BufferedReader(
             new FileReader("a.txt"))) {
    return r.readLine();
}
\end{minted}
\end{frame}
